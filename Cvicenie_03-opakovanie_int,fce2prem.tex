

\section{Opakovanie integračného počtu \\ Úvod do reálnej funkcie dvoch reálnych premenných}


\begin{enumerate}

\item Vypočítajte nasledujúce neurčité integrály použitím základných vlastností neurčitého integrálu, základných integračných vzorcov a vlastností elementárnych funkcií
\begin{enumerate}
\item{$ \int \frac{x}{\sin^2(x)} \,dx$} \hspace{\fill} [$2-x\cotg(x)+\ln(|\sin(x)|)+c$]
\item{$ \int e^x\cos(x) \,dx$} \hspace{\fill} [$\frac{e^x}{2}(\sin(x)+\cos(x)) +c$]
\item{$ \int \frac{\ln(x)}{x^2} \,dx$} \hspace{\fill} [$-\frac{1}{x}(\ln(x)+1) +c$]
\item{$ \int \frac{\ln^2(x)}{\sqrt{x}} \,dx$} \hspace{\fill} [$\sqrt{x}(2\ln^2(x)-8\ln(x)+16) +c$]
\item{$ \int x e^{-x} \,dx$} \hspace{\fill} [$e^{-x}(-x-1) +c$]
\item{$ \int \cos(\ln(x)) \,dx$} \hspace{\fill} [$\frac{x}{2}(\cos(\ln(x))+\sin(\ln(x)))+c$]

\item{$ \int \frac{\sin^2(x)}{\cos^4(x)} \,dx$} \hspace{\fill} [$\frac{1}{3}\tan^3(x)+c$]

\item{$ \int \frac{1}{x\ln(x)} \,dx$} \hspace{\fill} [$\ln(|\ln(x)|)+c$]

\item{$ \int \frac{1}{x\ln(x)\ln(\ln(x))} \,dx$} \hspace{\fill} [$\ln(|\ln(|\ln(x)|)|)+c$]

\item{$ \int \sin^5(x) \,dx$} \hspace{\fill} [$-\cos(x)+\frac{2}{3}\cos^3(x)-\frac{1}{5}\cos^5(x)+c$]
\item{$ \int \frac{\sin(2x)}{1+\sin^4(x)} \,dx$} \hspace{\fill} [$\arctan(\sin^2(x))+c$]
\end{enumerate}


\item Vypočítajte určité integrály
\begin{enumerate}
\item{$ \int \limits_1^2 \frac{x}{(x^2+1)^{\frac{3}{3}}} \,dx$} \hspace{\fill} [$\frac{-1}{\sqrt{5}}+\frac{1}{\sqrt{2}}$]
\item{$ \int \limits_0^{\frac{\pi}{2}} \frac{\sin^3(x)}{1+\cos^2(x)} \,dx$} \hspace{\fill} [$-1+\frac{\pi}{2}$]
\item{$ \int \limits_0^{\frac{\pi}{2}} \sin^3(x)\cos^2(x) \,dx$} \hspace{\fill} [$\frac{1}{3}-\frac{1}{5} +c$]
\item{$ \int \limits_4^5 \frac{\sqrt{x-4}}{1+\sqrt{x-4}} \,dx$} \hspace{\fill} [$-1+2\ln(2)$]
\item{$ \int \limits_{\ln(2)}^{\ln(3)} \frac{e^x}{e^{2x}-1} \,dx$} \hspace{\fill} [$\frac{1}{2}(\ln(\frac{1}{2})-\ln(\frac{1}{3}))$]
\end{enumerate}


\item Vypočítajte obsah plochy ohraničenej krivkou
\begin{enumerate}
\item{$ y=x^2$ a osou $x$ pre $x \in <-3,3>$} \hspace{\fill} [$18$]
\item{$ y=\frac{2}{1+x^2}$, $y=x^2$} \hspace{\fill} [$\pi-\frac{2}{3}$]
\item{$ y=x^3+x^2-6x$ a osou $x$ pre $x \in <-3,3>$} \hspace{\fill} [$28\frac{2}{3}$]
\item{$x=r\cos(t)$, $y=r\sin(t)$ pre $t \in <0,\pi>$} \hspace{\fill} [$\frac{\pi}{2}r^2$]
\end{enumerate}


\item Vypočítajte objem
\begin{enumerate}
\item{kuželu, ktorý vznikne rotáciou $y=\frac{1}{2}x-1$ okolo osy $x$ pre $x \in <2,6>$} \hspace{\fill} [$\pi\frac{16}{3}$]
\item{plochy medzi krivkami $ y=x^2+1$, $y=0$, $x=1$, $x=0$ okolo osy $y$} \hspace{\fill} [$\frac{3}{2}\pi$]
\end{enumerate}


\item Vypočítajte povrch gule (koule) o polomere $r$ kde je polkružnica daná 
\begin{enumerate}
\item{$y=\sqrt{r^2-x^2}$} \hspace{\fill} [$4\pi r^2$]
\end{enumerate}

\end{enumerate}