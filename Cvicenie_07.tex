\section{Lokální extrémy funkce dvou reálných proměnných}

\begin{enumerate}

\item Mějme funkci $f:\R \times \R \rightarrow \R$. Řekneme, že funkce $f$ má v bodě $A$ ostré lokální minimum (resp. maximum), jestliže $\exists$ ryzí okolí $\bar{O}(A)$ bodu $A[x_0,y_0]$ takové že $\forall X[x,y]  \in \bar{O}(A)$: $f(x,y) > f(x_0,y_0)$ (resp. $f(x,y) < f(x_0,y_0)$)

\item nutnou podmínkou existence lokálního extrému je, aby byly obě parciální derivace nulové, tedy $f^\prime_x (A) = f^\prime_y (A) = 0$ (takové body nazveme stacionární) nebo, že alespoň jedna z derivací neexistuje. 

\item vektoru $\vec{x}$ lze přiřadit číslo pomocí kvadratické formy: 
$$
\vec{x} \rightarrow \kappa = a_{11}\d x^2 + (a_{12} + a_{21}) \d x \d y + a_{22}\d y^2
$$
kvadratické formě přísluší její matice 
$
\left[
\begin{matrix}
a_{11} & a_{12} \\
a_{21} & a_{22}
\end{matrix}
\right]
$

takovou formou je např. nám již známý druhý diferenciál
$$
\d^2f_A = f^{\prime\prime}_{xx}(A) \d x^2
        +2f^{\prime\prime}_{xy}(A) \d x \d y
        + f^{\prime\prime}_{yy}(A) \d y^2
$$
a maticí této kvadratické formy je tzv. Hessova matice
$
H=
\left[
\begin{matrix}
f^{\prime\prime}_{xx}(A) & f^{\prime\prime}_{xy}(A) \\
f^{\prime\prime}_{xy}(A) & f^{\prime\prime}_{yy}(A) 
\end{matrix}
\right]
$
jejíž determinant je
$
det(H)=f^{\prime\prime}_{xx}(A) f^{\prime\prime}_{yy}(A)
     - \left(f^{\prime\prime}_{xy}(A)\right)^2
$

\item Postačující podmínkou existence lokálního extrému je definitnost kvadratické formy:
\begin{enumerate}
\item je-li kvadratická forma positivně definitní, pak je daný stacionární bod lokálním minimem
$$
\forall \vec{x} \neq \vec{o} :
\d^2f_A > 0 
\Rightarrow \exists \mathrm{lok.min.}
$$
Positivní definitnost lze ověřit pomocí Sylvesterova kriteria: 
$$
f^{\prime\prime}_{xx}(A) > 0; det(H) > 0
\Rightarrow \mathrm{pos.def.}
$$

\item je-li kvadratická forma negativně definitní, pak je daný stacionární bod lokálním maximem
$$
\forall \vec{x} \neq \vec{o} :
\d^2f_A < 0 
\Rightarrow \exists \mathrm{lok.max.}
$$
Negativní definitnost lze ověřit pomocí Sylvesterova kriteria: 
$$
f^{\prime\prime}_{xx}(A) < 0; det(H) > 0
\Rightarrow \mathrm{neg.def.}
$$

\end{enumerate}
Postačující podmínkou neexistence lokálního extrému (tj. existence sedlového bodu) je indefinitnost kvadratické formy
$$
\exists \vec{x_1},\vec{x_2} :
\d^2f_A(x_1) > 0  \vee \d^2f_A(x_2) < 0 
\Rightarrow \nexists \mathrm{extrem}
$$
Sylvestrovo kriterium říká, že kvadratická forma je indefinitní, není-li positivně ani negativně semidefinitní (tj. nejsou-li nerovnosti splněny ani v případě neostrých nerovností)

V případě semidefinitnosti (tj. pokud jsou nerovnosti neostré) nelze o extrému rozhodnout, pro takové případy existují složitější podmínky (ne)existence lokálních extrémů, těmi se však zabývat nebudeme.

\end{enumerate}