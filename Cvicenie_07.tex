\section{Lokální extrémy funkce dvou reálných proměnných}

\begin{enumerate}

\item Mějme funkci $f:\R \times \R \rightarrow \R$. Řekneme, že funkce $f$ má v bodě $A$ ostré lokální minimum (resp. maximum), jestliže $\exists$ ryzí okolí $\bar{O}(A)$ bodu $A[x_0,y_0]$ takové že $\forall X[x,y]  \in \bar{O}(A)$: $f(x,y) > f(x_0,y_0)$ (resp. $f(x,y) < f(x_0,y_0)$)

\item nutnou podmínkou existence lokálního extrému je, aby byly obě parciální derivace nulové, tedy $f^\prime_x (A) = f^\prime_y (A) = 0$ (takové body nazveme stacionární) nebo, že alespoň jedna z derivací neexistuje. 

\item vektoru $\vec{x}$ lze přiřadit číslo pomocí kvadratické formy: 
$$
\vec{x} \rightarrow \kappa = a_{11}\d x^2 + (a_{12} + a_{21}) \d x \d y + a_{22}\d y^2
$$
kvadratické formě přísluší její matice 
$
\left[
\begin{matrix}
a_{11} & a_{12} \\
a_{21} & a_{22}
\end{matrix}
\right]
$

takovou formou je např. nám již známý druhý diferenciál
$$
\d^2f_A = f^{\prime\prime}_{xx}(A) \d x^2
        +2f^{\prime\prime}_{xy}(A) \d x \d y
        + f^{\prime\prime}_{yy}(A) \d y^2
$$
a maticí této kvadratické formy je tzv. Hessova matice
$
H=
\left[
\begin{matrix}
f^{\prime\prime}_{xx}(A) & f^{\prime\prime}_{xy}(A) \\
f^{\prime\prime}_{xy}(A) & f^{\prime\prime}_{yy}(A) 
\end{matrix}
\right]
$
jejíž determinant je
$
det(H)=f^{\prime\prime}_{xx}(A) f^{\prime\prime}_{yy}(A)
     - \left(f^{\prime\prime}_{xy}(A)\right)^2
$

\item Postačující podmínkou existence lokálního extrému je definitnost kvadratické formy:
\begin{enumerate}
\item je-li kvadratická forma positivně definitní, pak je daný stacionární bod lokálním minimem
$$
\forall \vec{x} \neq \vec{o} :
\d^2f_A > 0 
\Rightarrow \exists \mathrm{lok.min.}
$$
Positivní definitnost lze ověřit pomocí Sylvesterova kriteria: 
$$
f^{\prime\prime}_{xx}(A) > 0; det(H) > 0
\Rightarrow \mathrm{pos.def.}
$$

\item je-li kvadratická forma negativně definitní, pak je daný stacionární bod lokálním maximem
$$
\forall \vec{x} \neq \vec{o} :
\d^2f_A < 0 
\Rightarrow \exists \mathrm{lok.max.}
$$
Negativní definitnost lze ověřit pomocí Sylvesterova kriteria: 
$$
f^{\prime\prime}_{xx}(A) < 0; det(H) > 0
\Rightarrow \mathrm{neg.def.}
$$

\end{enumerate}
Postačující podmínkou neexistence lokálního extrému (tj. existence sedlového bodu) je indefinitnost kvadratické formy
$$
\exists \vec{x_1},\vec{x_2} :
\d^2f_A(x_1) > 0  \vee \d^2f_A(x_2) < 0 
\Rightarrow \nexists \mathrm{extrem}
$$
Sylvestrovo kriterium říká, že kvadratická forma je indefinitní, není-li positivně ani negativně semidefinitní (tj. nejsou-li nerovnosti splněny ani v případě neostrých nerovností)

V případě semidefinitnosti (tj. pokud jsou nerovnosti neostré) nelze o extrému rozhodnout, pro takové případy existují složitější podmínky (ne)existence lokálních extrémů, těmi se však zabývat nebudeme.


\item Praktický (algoritmický) prístup k Sylvestrovmu kritériu:
\begin{enumerate}
    \item[a)] $f''_{xx}(A) < 0; det(H(A)) > 0 \implies $ ostré maximum v $A$
    \item[b)] $f''_{xx}(A) > 0; det(H(A)) > 0 \implies $ ostré minimum v $A$
    \item[c)] $det(H(A)) < 0 \implies $ sedlo v $A$ (není extrém)
    \item[d)] $det(H(A)) = 0 \implies $ buď lokálne minimum, maximum alebo sedlo, ale musíme použít definici :(.
    \item[e)] $f''_{xx}(A)=0$ a~$det(H(A)) = 0 \implies $ buď lokálne minimum, maximum alebo sedlo, ale musíme použít definici :(.
    
\end{enumerate}

\item Nalezněte lokální extrémy následujících funkcí:
\begin{enumerate}
    \item[$a_K)$]{$f(x,y)=x \arctan(y+2)$}
    \item[$b_K)$]{$f(x,y)=x^3y+x+y-\frac{1}{216y^2}$}
    \item[$c_K)$]{$f(x,y)=x^2y-xy^2-e^{y+16}$}
    \item[$d_K)$]{$f(x,y)=x^3-xy^2+\frac{9}{y^2}$}
\end{enumerate}
\item Řešení předchozího příkladu:
\begin{enumerate}
    \item[$a_K)$]{$f_x'(x,y)=\arctan(y+2)$\\
    $f_y'(x,y)=x\frac{1}{1+(y+2)^2}$\\
    Stacionární bod je zřejmý $A=[0,-2]$. \\
    $f_{xx}''(x,y)=0$\\
    $f_{yy}''(x,y)=x\frac{2(y+2)}{(1+(y+2)^2)^2}$\\
    $f_{xy}''(x,y)=\frac{1}{1+(y+2)^2}$\\
    $det(H(x,y))=0\cdot x\frac{2(y+2)}{(1+(y+2)^2)^2} -\frac{1}{(1+(y+2)^2)^2}$\\
    Ze Sylvestrova kritéria vidíme, že $H(A)=-1<0$, tedy bod $A=[0,-2]$ je sedlovým bodem.}
    \item[$c_K)$]{
    $f_x'(x,y)=2xy-y^2$\\
    $f_y'(x,y)=x^2-2xy-e^{y+16}$\\
    Stacionární body $A=[-e^8,0]$ a $B=[e^8,0]$\\
    $f_{xx}''(x,y)=2y$\\
    $f_{yy}''(x,y)=-2x-e^{y+16}$\\
    $f_{xy}''(x,y)=2x-2y$\\
    $det(H(x,y))=2y (-2x-e^{y+16})-(2x-2y)^2=-4x^2$\\
    $det(H(B))<0$ \\
    $det(H(A))<0$ \\
    Ze Sylvestrova kritéria vidíme, že $H(B)<0$ , $H(A)<$, tedy body $A,B$ jsou sedla.
    }
    \item[$d_K)$]{$f_x'(x,y)=3x^2-y^2$\\
    $f_y'(x,y)=-2xy-\frac{18}{y^3}$\\
    Stacionární bod získáme dosazením vztahu $x=-\frac{9}{y^4}$ (vztah získaný z $f_y'(x,y)=0$) do $3x^2=y^2$ ($f_x'(x,y)=0$). Stacionárnímy body teda jsou $A=[-1,\sqrt{3}]$ a $B=[-1,-\sqrt{3}]$\\
    $f_{xx}''(x,y)=6x$\\
    $f_{yy}''(x,y)=-2x+\frac{3^3 2}{y^4}$\\
    $f_{xy}''(x,y)=-2y$\\
    $det(H(x,y))=6x \cdot (-2x+\frac{3^3 2}{y^4}) -4y^2$\\
    $det(H(A))=det(H(B))=-60$
    Ze Sylvestrova kritéria vidíme, že $H(A)<0$ i $H(B)<0$, tedy body $A,B$ jsou sedla.}
    
\end{enumerate}

\end{enumerate}